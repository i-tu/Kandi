	\documentclass[finnish,12pt]{article}
	\usepackage{aaltothesis}
  
	%% Use this if you run latex and use eps-format pictures
	\usepackage[dvips]{graphicx}
	%\usepackage[pdftex]{graphicx} 

	\usepackage[pdfpagemode=None,colorlinks=true,urlcolor=red,linkcolor=blue,citecolor=black,pdfstartview=FitH]{hyperref}

	%% Use this if you do not like hyperref package - this defines url environment and formats it correctly
	%\usepackage{url}

	% Math stuff
	\usepackage{amsfonts,amssymb,amsbsy}

  \usepackage{lmodern}
	
	% Layout
	\setlength{\hoffset}{-1in}
	\setlength{\oddsidemargin}{35mm}
	\setlength{\evensidemargin}{25mm}
	\setlength{\textwidth}{15cm}
	\setlength{\voffset}{-1in}
	\setlength{\headsep}{7mm}
	\setlength{\headheight}{1em}
	\setlength{\topmargin}{25mm-\headheight-\headsep}
	\setlength{\textheight}{23cm}

	\begin{document}

	\university{aalto-yliopisto}
	\school{Sähkötekniikan korkeakoulu}

	\faculty{Automaatio- ja systeemitekniikka}
	\degreeprogram{Automaatio- ja systeemitekniikka}
	\univdegree{BSc}
	\author{Ian Tuomi}
	\thesistitle{Automaatiokaavioiden generointi}
	\place{Espoo}
	\date{5.12.2012}
	\supervisor{super visor}
	\instructor{DI Mika Strömman}
	\uselogo{red}{!}{tkk}
	\makecoverpage

	\keywords{PLC ß}
	%% Tiivistelmän tekstiosa
	\begin{abstractpage}[finnish]
  
LYHYT TIIVISTELMÄ
  
	\end{abstractpage}

	\newpage

	\mysection{Esipuhe}

Haluan kiittää seuraavia ihmisiä joita ilman en olisi onnistunut tässä työssä
Kandiryhmäni jäseniä Lauri Vepsäläistä, Heikki 'Taffis' Tahvanaista, Panu Kauppista, Peter 'Bembu'  Kronströmiä ja Lauri Andleria.
Kandikurssin järjestäjää 
Mika Strömmania hyvistä huomioista,  asiallisesta palautteesta ja skarpista ohjauksesta.
Petri Kokkoa ja Pekka Niemistä ideoinnista ja suunnannäyttämisestä.

Aalto- yliopiston henkilökuntaa hyvästä työstä.
Vanhempiani tuesta ja kannustuksesta.
\\

	\vspace{5cm}

Otaniemi, 16.3.2010

	\vspace{5mm}
	{\hfill Ian Tuomi \hspace{1cm}}

	\newpage

	\addcontentsline{Sisällysluettelo}
	\tableofcontents


	%% Symbolit ja lyhenteet
	\mysection{Symbolit ja lyhenteet}
	\subsection*{Symbolit}

	\begin{tabular}{ll}
	%$|a_{ij}|^2$, $|a_i|^2$ & probability of two electrons having momenta
	%    $\boldsymbol p_i$ and $\boldsymbol p_j$ ($\boldsymbol p_i$ for $|a_i|^2$) \\
	%                 & at any given instant \\
	$\mathbf{B}$  & magneettivuon tiheys  \\
	$c$              & valon nopeus tyhjössä $\approx 3\times10^8$ [m/s]\\
	%$p$              & magnitude of momentum \\
	%$\boldsymbol p$, $\boldsymbol p_i$, $\boldsymbol p_i^{'}$  & momentum vector \\
	%$p$              & magnitude of momentum \\
	%$\boldsymbol p$, $\boldsymbol p_i$, $\boldsymbol p_i^{'}$  & momentum vector \\
	%$\boldsymbol P$  &  \\
	%$p_{\mathrm{F}}$ & Fermi momentum \\
	$\omega_{\mathrm{D}}$    & Debye-taajuus \\
	$\omega_{\mathrm{latt}}$ & hilan keskimääräinen fononitaajuus \\
	$\uparrow$       & elektronin spinin suunta ylöspäin\\
	$\downarrow$     & elektronin spinin suunta alaspäin
	\end{tabular}

	\subsection*{Lyhenteet}

	\begin{tabular}{ll}
         & vaihtovirta \\
PLC      & Programmable Logic Circuit \\
           & (originally Analysis Program for Linear Active Circuits) \\
BCS        & Bardeen-Cooper-Schrieffer \\ %% tavuviiva - nimien välissä 
DC         & tasavirta \\
TEM        & transverse eletromagnetic
	\end{tabular}


	\cleardoublepage
	\storeinipagenumber
	\pagenumbering{arabic}
	\setcounter{page}{1}

	\section{Johdanto}
	\thispagestyle{empty}

Teollisuuden automaatiojärjestelmän suunnittelutyön tulos koostuu kaavioista - laajan järjestelmän tapauksessa jopa useista tuhansista. 
Kaavioiden piirtäminen ja ajantasaisena pitäminen on työlästä ilman automatisoituja suunnittelun apuvälineitä.
Tämän vuoksi kaaviot tyypillisesti generoidaan suunnittelijoiden laatimista kuvauksista, joiden muokkaamisesta suunnittelutyö pääasiallisesti koostuu.

Tämän tutkimuksen kohteena on järjestelmän ohjauslogiikan formaalin kuvauksen muuntaminen standardeita noudattavaksi logiikkakaavioksi.
Logiikkakaavio on määritelty yksiselitteisesti ja on tulkittavissa suunnatuksi graafiksi.

Logiikkakaavioiden kuvaukset on määritelty esimerkiksi standardeissa IEC~61131-3 ja IEC~61499.
Lisäksi suunnittelijoilla saattaa olla käytössä jokin muu, ei-standardi tapa kuvata järjestelmää.

Algoritmillinen graafipiirto on perusteellisesti tutkittu ala, jonka tuloksia voidaan käyttää kaikenlaisten kaavioiden piirtoon.
Esittelemme Sugiyaman esittämän hierarkisen lähestymistavan. jonka avulla lopullinen kaavio piirretään.

Tavoitteena on arvioida eri keinojen hyviä ja huonoja puolia sekä käytännön toteutusvaatimuksia.
Lisäksi tarkastellaan voisiko näitä ratkaisuja laajentaa muunlaisten kaavioiden tuotantoon.

	\clearpage
	\section{Tehdasmalli}

Tehdasmalli on kuvaus joka mahdollistaa tehtaan toteutuksen, käyttöönoton ja ylläpidon.
Se on tietomalli joka kuvaa laajuudestaan riippuen tehtaan toimintaa, sen prosessia, organisaatiota, ihmisiä ja näitten aktiviteetteja.

Tehdasmallilla on metamalli joka määrittelee millaisia objekteja tehdasmalli voi sisältää, mitä ominaisuuksia niillä voi tai täytyy olla, ja millaisia yhteyksiä objektien välillä on.
Tehdasmalli koostuu pohjimmiltaan tehdasobjekteista, niitten ominaisuuksista ja niitten välisistä relaatioista. 

Vaikka tehtaasta on vain yksi malli, on eri suunnittelutehtävillä erilaisia tarpeita.
Tehdasmallin tulee siis palvella kaikkia näkökulmia joita tarvitaan.  
Suunnitteluprosessin tuloksena syntyy järjestelmän malli.

Käytännössä tehdasmalli on dokumenttihotellin typpinen sovellus, joka koostuu 3d-mallista, näihin liittyvistä dokumenteista sekä käyttöliittymästä.

Malli voi olla joukko dokumentteja, tai suunnittelijoiden yhteinen tietokanta, ns. tehdasmalli, jonka avulla voidaan tuottaa järjestelmän dokumentointi.

		\subsection{Ontologia}



KUVA: Tehdasmallin abstraktiotasot. (Lähde: Automaatiosuunnittelun prosessimalli)

		\subsection{Tietokanta}

Tietokanta on perinteisesti linkitetty tietokanta. (lähde?)

Tehdasmallin voidaan sanoa olevan tietomalli.

Automaatiojärjestelmän näkymiä ovat
Minimivaatimukset kuvaukselle
Automaatiojärjestelmien ontologia
”Semanttinen kuvaus”

	\clearpage
	\section{PLC-ohjelmointi}

Varhainen tapa hallita tehtaiden prosesseja sähkölaitteiden avulla oli relelogiikka.
Jos esimerkiksi moottori käyttää liikaa virtaa, kytketään se pois päältä.
Tietylle syöttöavaruudelle tietty lähtöavaruus.
Kalliissa ja hankalassa relelogiikassa oli kuitenkin omat ongelmansa, ja siirryttiin PLC-ohjaukseen.
PLC- ohjaus vie vähemmän tilaa, on halvempaa ja helpommin uudelleenohjelmoitavissa.

PLCeitä usein ohjelmoidaan vanhasta tottumuksesta tikapuukaavioilla.
Näitten ymmärtäminen on intuitiivisesti vaikeampaa ja ne usein vaativat alan tuntemusta.
Logiikkadiagrammit ovat yritys pelkistää yksinkertaisimpaan mahdolliseen esitykseen monimutkainen logiikka joka on syöttöjen ja lähtöjen välillä.
Vaikka PLC saatetaan myöhemmin ohjelmoida tikapuukaavioilla, on silti helpompaa työskennellä ja ymmärtää ohjelman rakenne logiikkadiagrammien avulla.



(Formal methods in PLC programming, Frey, Litz)

Frey ja Litz vertailevat säätösuunnitteluprosessia ja vertailevat eri formaaleja prosesseja ja kategorisoivat PLC-ohjelmoinnin menetelmiä. 
Siitä huolimatta että PLC-ohjelmointi on vanha ja intuitiivinen ala, on selviä syitä suunnittelun formalisoitumiselle.
Säätö-ongelmat monimutkaistuvat ja vaadittu toteutusaika vähenee.
Vanhan suunnittelutyön hyväksikäyttö vaatii formalismien käyttöönoton.
PLC-ohjelmointijärjestelmien laatu- ja turvallisuusvaatimusten kasvaessa 



		\subsection{Logiikkakaaviot}
	
mcavinew, control system documentation

Funktioblokit kuvaavat toisistaan erillisiä itsenäisiä laskennallisia yksiköitä ja niitten relaatioita toisiinsa.
Funktioblokki voi tilastaan riippuen lähettää tapahtumia jotka vaikuttavat muitten blokkien toimintaan.
Funktioblokin tekemien laskutoimitusten tulos riippuu siis sen vasemmalta puolelta tulevista syöttötiedoista.
Laskennan tulokset lähetetään blokin oikealta puolelta.
Toisiinsa liitetyt funktioblokit muodostavat funktioblokkiverkon. 

IEC 61499 määrittelee XML-kuvauksen funktioblokkiverkoille. 

Although the IEC 61499 is not always seen as the ultimate solution, it is a useful and powerful basis for engineering tools that follow an abstract, model-driven approach and can generate the actual function blocks and their interconnections automatically from a high-level description of the required functionalities." -10.1109/MIE.2011.942175 (Sauter et. al. IEEE Industrial electronics magazine, p. 35)


		\subsection{Logiikkakaavioiden tuotanto}


Logiikkaa sisältävän järjestelmän suunnittelussa on kaksi vaihetta, jotka tulee molemmat ottaa huomioon.
Ensimmäinen on prosessin konseptualisointi ja toinen toteutus.
Hajottamalla prosessi kahteen voidaan prosessia kehittää ja siitä keskustella ilman syvällistä tuntemusta instrumenteista.
Kun päätökset on tehty abstraktilla tasolla, voidaan hankinnat räätälöidä sen mukaisesti ja tulos toteuttaa tehokkaasti.

Mentäessä konseptista toteutukseen voidaan nähdä että vähintään kahdenlaisia dokumentteja tarvitaan.
Toteuttava dokumentti saattaa olla ainoa, joka näkee päivänvalon. 

Yleisin tapa tuottaa kaavioita lienee niitten käsin tekeminen.
Valmiiden grafiikkakirjastojen käyttöä on tutkinut Kääriäinen, jonka mukaan työssä .
next generation systems…

http://mv-sirius.fh-offenburg.de/ecmIC/SensorActuatorObj-Worksheet/atr_2001.pdf

		\subsection{Suoritusjärjestys}
	
Automaatiokaavioiden automaattinen piirto voi tuottaa ongelmia suoritusjärjestystä mietittäessä. 
Millä tavoin?
Miten sen voi välttää?
Automaatiokaavioiden suoritusjärjestys on määritelty standardissa IEC 61131-3 ylhäältä alas, vasemmalta oikealle.
Tämä tarkoittaa sitä, että graafin uudelleenjärjestely johtaa vääränlaiseen kaavioon.
Tämä työ olettaa, että eri ohjelmien suoritusjärjestys niillä kohdin, kuin se on olennaista, on tiedossa, ja että kyseisiö
Automaatiokaavioiden automaattinen piirto voi tuottaa ongelmia suoritusjärjestystä mietittäessä. 
Millä tavoin?
Miten sen voi välttää?
Standardi: ylhäältä alas, vasemmalta oikealle.



	\clearpage
	\section{Hierarkinen graafinpiirto}

Hierarkisen lähestymistavan kerrostettujen suunnattujen graafien piirtämiseen esitti Sugiyama et. al vuonna 1981.
Menetelmää on kehitetty sen jälkeen eteenpäin samanlaisilla menetelmillä.
Hierarkinen lähestymistapa koostuu kolmesta vaiheesta: kerrostamisesta, risteämisten vähentämisestä ja poikittaisasettelusta.
Jos lähdegraafi sisältää takaisinkytkentöjä, vaatii

http://www.cs.brown.edu/~rt/gdhandbook/

Sitä on sen jälkeen kehitetty eteenpäin 
Sugiyama et al. ovat tutkineet ongelmaa [1]. Muut ovat sitä jatkokehitelleet [2]. Algoritmi selitetään.

Kerrostetut suunnatut graafit ovat luonteeltaan hierarkisia, kuten logiikkakaaviot suoritusjärjestyksensä puolesta.
Kerrostetun suunnatun graafin piirto on jaettu kolmeen vaiheeseen. Kerroksiin jako, lankojen risteämisten minimointi, x-koordinaatin määritely.

Jos graafissa on syklejä, eli graafi ei ole suunnattu, käännetään syklin aiheuttavien lankojen suunta ja tehdyt muutokset merkitään muistiin.

		\subsection{Kerrostus}

Kerrostuksen tulee täyttää tietyt ehdot.
Sen tulee olla kompakti.
Tämä tarkoittaa että sen leveys ja pituus ovat pieniä ja kerrosten välinen etäisyys on vakio.
Graafin pituuden alaraja on maksimimäärä lankoja alkunoodista loppunoodiin.
On olemassa menetelmä, jonka avulla graafin pituus voidaan minimoida leveyden kustannuksella tai siitä voidaan tehdä mahdollisimman kapea pituuden kustannuksella.

Leveyden ja pituuden minimointi taas on NP- täydellinen ongelma, seuraavasta syystä.
Jos jokainen solmi suunnatussa kerrostetussa verkossa G esittää yhden aikayksikön työtä yhdessä multiprosessorin prosessorissa, lanka (u, v) esittää rajoitetta, jonka mukaan suorituksen u täytyy edeltää suoritusta. (Battista, 273)
Tällöin voidaan löytää isomorfismi multiprosessoriajastusongelmalle, jossa tehtävänä on jakaa tehtävät prosessorien kesken siten, että tehtävä tulee suoritettua ajassa H.
Koska multiprosessoriajastusongelma on NP-täydellinen (M.R. Garey and D.S. Johnson, Computers and Intractibility: A guide to the theory of NP-completeness. 1979) , seuraa, että kerrostusongelmakin on.
Yhteys multiprosessoriongelmaan on kiinnostava käsittelemämme sovelluksen, PLC-ohjelmoinnin kannalta.
Multiprosessoriongelmaa varten tehty algoritmi, Coffman-Graham-kerrostusmenetelmä (Coffman, Optimal scheduling for two processor systems, 1972) tarjoaa ratkaisun tähän ongelmaan.
Se palauttaa graafin, jolla on jokin minimileveys, alkuperäisessä toteutuksessa prosessorien määrä, omassa tapauksessamme arkin korkeus. 
Algoritmin tavoitteena on varmistaa, että 

		\subsection{Risteyksien vähentäminen}

Risteyksien vähentäminen ei perustu solmujen tarkkoihin sijainteihin, vaan niitten keskinäiseen järjestykseen.
Ongelma on siis luonteeltaan kombinatorinen eikä geometrinen, mikä huomattavasti yksinkertaistaa ongelman ratkaisemista. 
Tästä helpotuksesta huolimatta kyseessä on NP-täydellinen ongelma (M.R. Garey ”Crossing problem is NP-complete”) siinäkin tapauksessa että kerroksia no vain kaksi.

Lähestymistapoja risteyksien vähentämiseen on vierekkäisten noodien paikkojen vaihtelu keskenään, sekä mediaanimenetelmä, jossa noodi asetetaan paikalle, joka on siihen yhtyneiden solmujen puolivälissä.

Monet tutkijat ovat Tähän se väikkäri myös! (P. Eades, ”Heuristics for Reducing crossings in 2-layered networks”, E. Mäkinen, ”experiments of drawing 2-level hierarchical graphs”, M. Jünger, ”2-layer straightline crossing minimization: Perf…” ) tutkineet eri risteysalgoritmeja. Yhtä selkeästi parasta menetelmää ei ole löytynyt. Sen sijaan paras lähestymistapa on osoittautunut algoritmeja yhdisteleväksi hybridilähestymiseksi.
(Joku et al on) saavuttanut hyviä tuloksia iteroimalla joka iteraation yhteydessä ensin uudelleenjärjestää solmut mediaanimenetelmällä, tekee paikallisia transpositioita ja lopulta pitää parhaan tuloksen seuraavaa iterointia varten.


Optimizing Automatic Layout for Data Flow Diagrams, 2011, Diploma thesis, Christian-Albrechts-Universität zu Kiel, Department of Computer Science

		\subsection{Poikittainen asettelu}

Solmujen poikittainen asettelun avulla solmujen välisten lankojen käännösten määrää pyritään vähentämään, jotta kaavion luettavuus paranee. Risteyksien vähentämisvaiheessa saavutettu järjestys pyritään säilyttämään.

	\clearpage
	\section{Menetelmät ja esimerkki}
		\subsection{Kaavioiden tuottaminen}

Käytännön toteutuksia hierarkisesta graafinpiirtomenetelmästä on useita (viittaa: graphviz, tulip, kieler)’
An open graph visualization system and its applications to software engineering (2000) 

Automatic generation of PLC automation projects

from component-based models
Elisabet Estévez
Käytännönläheinen selostus siitä, kuinka aiempaa teoriaa sovelletaan.

		\subsection{Kaavioiden tuottaminen}
			\subsubsection{Vertaillut vaihtoehdot}
	
Funktioblokkien editointiin on olemassa useita työkaluja.
Rockwell Automationin ilmainen funktioblokkieditori, FBDK on tutkimuksessa laajassa käytössä.
FBDK ei kuitenkaan tarjoa tapoja muuttaa kaavion asettelua algoritmillisesti.
FBench on avoimen lähdekoodin standardimukainen ohjelmisto, joka kuitenkin kärsii samoista puutteista kuin FBDK.
PROFACTORin kehittämä 4DIAC tarjoaa standardeja noudattavan funktioblokkieditorin.
Se ei kuitenkaan mahdollista blokkien välisten lankojen reitittämistä.
ISaGRAF ja NxtControl ovat kaupallisesti tuotettuja ohjelmistoja joita ei valitettavasti voitu työtä varten arvioida.

			\subsubsection{KIELER}

Koeohjelmistoksi valittiin KIELER, Kielin yliopistossa kehitetty yleiseen graafinpiirtoon tuotettu ohjelmisto, jolle on kehitetty IEC 61499-standardin mukaisten funktioblokkidiagrammien piirto-ohjelmisto CAKeFEED. 
Kerrotaan tarkemmin ohjelmistosta jolla koe on suoritettu

		\subsection{Tulosten arviointiperusteet}

Lankojen risteämismäärien minimointi !
Ortogonaalisuus!
Graph Layout Aesthetics in UML Diagrams: User Preferences

The Aesthetics of Graph Visualization

	\clearpage
	\section{Tulokset}
	
Olemme osoittaneet yhteyden graafiteorian suunnatuilla puilla ja logiikkadiagrammeilla ja näyttäneet miten graafipiirtoalgoritmeja voidaan käyttää automaatiokaavioiden generointiin.
Tapa X on kivempi kuin tapa Y.

	\section{Yhteenveto ja tulevaisuudennäkymät}
	
Siistejä juttuja, joo. Ollaan saavutettu tavoite A. Tässä hommassa kokonaisuutena on vielä haasteita. Olisipa hienoa saada selville asia Z

Automaatiosuunnittelu on siirtynyt yhä enemmän tietopohjaisesta tutkimuksesta

RDF:ää voidaan käyttää tehtaan PI-kaavion miettimiseen.
Graph theory - An RDF model is a labeled, directed multi-graph


\clearpage
\addcontentsline{toc}{section}{Viitteet}

\begin{thebibliography}{99}
	\bibitem{Kauranen} Kauranen,\ I., Mustakallio,\ M. ja Palmgren,\ V.
	  \textit{Tutkimusraportin kirjoittamisen opas opinnäytetyön
		tekijöille.}  Espoo, Teknillinen korkeakoulu, 2006.

	\bibitem{Itkonen} Itkonen,\ M. \textit{Typografian käsikirja.} 3.\
	  painos.  Helsinki, RPS-yhtiöt, 2007.

	\bibitem{Koblitz} Koblitz,\ N. \textit{A Course in Number Theory and
		Cryptography. Graduate Texts in Mathematics 114.}  2.\ painos. New
	  York, Springer, 1994.

	\end{thebibliography}

\end{document}


