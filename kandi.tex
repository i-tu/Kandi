	\documentclass[finnish,12pt]{article}
	\usepackage{aaltothesis}
        \usepackage[T1]{inputenc}
	%% Use this if you run latex and use eps-format pictures
	\usepackage[dvips]{graphicx}
	%\usepackage[pdftex]{graphicx} 

	\usepackage[pdfpagemode=None,colorlinks=true,urlcolor=red,linkcolor=blue,citecolor=black,pdfstartview=FitH]{hyperref}

	%% Use this if you do not like hyperref package - this defines url environment and formats it correctly
	%\usepackage{url}

	% Math stuff
	\usepackage{amsfonts,amssymb,amsbsy}

	\usepackage{setspace}
	\onehalfspacing
	
  \usepackage{lmodern}
	
	% Layout
	\setlength{\hoffset}{-1in}
	\setlength{\oddsidemargin}{35mm}
	\setlength{\evensidemargin}{25mm}
	\setlength{\textwidth}{15cm}
	\setlength{\voffset}{-1in}
	\setlength{\headsep}{7mm}
	\setlength{\headheight}{1em}
	\setlength{\topmargin}{25mm-\headheight-\headsep}
	\setlength{\textheight}{23cm}

	
	\begin{document}
	
	\bibliographystyle{unsrt}

	\university{Aalto-yliopisto}
	\school{Sähkötekniikan korkeakoulu}

	\faculty{Automaatio- ja systeemitekniikan laitos}
	\degreeprogram{Automaatio- ja systeemitekniikka}
	\univdegree{BSc}
	\author{Ian Tuomi}
	\thesistitle{Automaatiokaavioiden generointi}
	\place{Espoo}
	\date{5.12.2012}
	\supervisor{DI Marek Matusiak}
	\instructor{DI Mika Strömman}
	\uselogo{red}{!}{tkk}
	\makecoverpage

	\keywords{PLC ß JOOJJEEE}
	%% Tiivistelmän tekstiosa
	\begin{abstractpage}[finnish]
  
Tämän tutkimuksen kohteena on järjestelmän ohjauslogiikan formaalin kuvauksen muuntaminen standardeita noudattavaksi logiikkakaavioksi.
  
	\end{abstractpage}

	\newpage
	\vspace{10cm}
	\mysection{Esipuhe}

Haluan kiittää seuraavia ihmisiä joita ilman en olisi onnistunut tässä työssä.

Ohjaajaani Mika Strömmania hyvistä huomioista,  asiallisesta palautteesta ja skarpista ohjauksesta.

Kandiryhmäni jäseniä Lauri Vepsäläistä, Heikki 'Taffis' Tahvanaista, Panu Kauppista, Peter 'Bembu'  Kronströmiä ja Lauri Andleria.

Kandikurssin järjestäjää 

Petri Kokkoa ja Pekka Niemistä ideoinnista ja suunnannäyttämisestä.

Aalto- yliopiston henkilökuntaa hyvästä työstä.
Vanhempiani tuesta ja kannustuksesta.
\\

	\vspace{5cm}

Otaniemi, 16.3.2010

	\vspace{5mm}
	{\hfill Ian Tuomi \hspace{1cm}}

	\newpage

	\addcontentsline{Sisällysluettelo}

	\tableofcontents


	%% Symbolit ja lyhenteet
	\mysection{Lyhenteet}

	\begin{tabular}{ll}
DCS        & Distributed Control System \\
	      & Hajautettu ohjausjärjestelmä
PLC      & Programmable Logic Circuit \\
	      & Ohjelmoitava logiikkapiiri
	\end{tabular}

	\cleardoublepage
	\storeinipagenumber
	\pagenumbering{arabic}
	\setcounter{page}{1}

	\section{Johdanto}
	\thispagestyle{empty}

Teollisuuden automaatiojärjestelmän suunnittelutyön tulos koostuu kaavioista. Laajan järjestelmän tapauksessa niitä voi olla jopa useita tuhansia. Kaaviot muodostavat tehtaan kuvauksen, tehdasmallin, joka mahdollistaa sen toteutuksen, käyttöönoton ja ylläpidon. Kaavioiden piirtäminen käsin on kuitenkin suurissa projekteissa työlästä ja tehotonta. Nykyaikaisessa tehdassuunnittelussa ongelma korostuu, kun ratkaisut ovat entistä monimutkaisempia ja vaadittu toteutusaika lyhenee. Myös laatuvaatimukset ovat tiukentuneet.

Tämän vuoksi suunnitteluprosessia on pyritty tehostamaan automatisoiduilla suunnittelun apuvälineillä. Suunnittelumetodiikat eivät ole kuitenkaan ilmeisistä standardoimiseduista huolimatta yhtenäistyneet. Vaikka lähestymistavat ovat vaihtelevia, on tehdassuunnittelussa kuitenkin nähtävissä yhtenäinen siirtymä kohti tehtaan tietomalleja. Tietomalliin perustuva lähestymistapa vaatii tietomallin rajoitukset määrittelevän metamallin. Tämä ohjaa ja määrittää suunnittelun kohteen lisäksi myös suunnittelutyön kulkua.

Suunnittelujärjestelmään syntyy tällöin projektin edetessä tehtaan tietomalli johon tehtaan kaikki sähköistys ja instrumentointi automaatiojärjestelmineen ja kenttälaitteineen on merkitty. Se sisältää kuvauksen jonka avulla voidaan tuottaa tehtaan lopullinen dokumentointi.

Suunnittelutyö koostuu tällöin tietomallin muokkaamisesta. Tämä mahdollistaa tehokkaiden suunnittelun apuvälineiden kehittämisen. Erilaiset kaaviot voidaan tuottaa suoraan tietomallin sisältävästä tietokannasta joiden muokkaamisesta suunnittelutyö pääasiallisesti koostuu.

Tämän tutkimuksen kohteena on tehtaan hajautetun ohjausjärjestelmän logiikan tietomallin muuntaminen standardeita noudattavaksi logiikkakaavioksi. Logiikkakaaviot ovat tapa pelkistää syöttöjen ja lähtöjen välillä vallitseva monimutkainen logiikka yksinkertaisimmaksi mahdolliseksi esitykseksi. Logiikkakaavio on määritelty yksiselitteisesti ja on tulkittavissa suunnatuksi graafiksi, jolloin algoritmillisen graafinpiirron tuloksia voidaan hyödyntää.

Algoritmillinen graafipiirto on perusteellisesti tutkittu ala, jonka tuloksia voidaan käyttää kaikenlaisten suunnattujen kaavioiden piirtoon. Esittelemme Sugiyaman et. al. \cite{RefWorks:9} esittämään hierarkiseen lähestymistapaan perustuvan algoritmin jonka avulla lopullinen kaavio piirretään.

Lisäksi esitellään erilaisia olemassaolevia toteutuksia ja 


%%%%%%%%%%%%%%%%%%%%%%%%%%%%%%%%%%%%
	\clearpage
	\section{Suunnittelutyön mallit}
%%%%%%%%%%%%%%%%%%%%%%%%%%%%%%%%%%%%

Työyhteisön kesken jaetut mallit ovat tärkeitä yhteistyön apuvälineitä. Ne hahmottavat projektin kulkua, työn sisältöä ja vaatimuksia sekä tarjoavat projektin työntekijöille yhteisen terminologian. Monimutkaisissa projekteissa kuten tehdassuunnittelussa, on mallien merkitys suuri. Ilman systemaattista suunnittelun mallia kommunikointi vaikeutuu ja suunnittelutyö hidastuu.

Kun suunnittelutyön sisältö määritellään tarkasti, suunnittelijat voivat tehdä yhteistyötä tehokkaasti ja toisaalta keskittyä omaan suunnittelualueeseensa. Lisäksi tarkka malli mahdollistaa suunnittelun apuvälineiden kehittämisen.

		\subsection{Mallien ontologia}

Mikään yksi malli ei pysty kattamaan kaikkia suunniteltavan järjestelmän näkymiä. Sen sijaan sekä tehdasmalli että suunnitteluprosessi koostuvat useista malleista, jotka yhdessä muodostavat kokonaisuuden.

Kruchten [3] esittääkin, että useimmat monimutkaiset suunnittelutyöt vaativat vähintään neljää eri näkymää: loogista näkymää, prosessinäkymää, kehitysnäkymää sekä fyysistä näkymää. Tämä järjestely onkin jo käytössä automaatiosuunnittelussa.

Tehdasmallilla on metamalli joka määrittelee millaisia objekteja tehdasmalli voi sisältää, mitä ominaisuuksia niillä voi tai täytyy olla ja millaisia yhteyksiä objektien välillä on. Metamallin loogiset riippuvuudet määrittelevät suunnittelun työtapoja- ja järjestystä.

Metamalli on kuvaus siitä, millaisia valiidit tehdasmallit ovat. Se kuvaa millaiset tietomallit ovat tehdasmalleja. Se määrää samalla suunnittelujärjestelmän laajuuden ja järkevät toteutustavat. Abstraktoimalla ja käsitteellistämällä tehdas toisaalta suljetaan pois mahdollisia tehtaan abstraktioita ja toisaalta mahdollistetaan pitkälle vietyyn formalismiin perustuvat työkalut ja suunnittelukäytännöt.

Projektin eri työntekijöillä projektipäälliköstä perussuunnittelijaan on erilaisia tarpeita, joita varten on erilaisia malleja. Suunnittelutyön mallit on valittu siten, etteivät ne ole ristiriidassa ja muodostavat oman suljetun kokonaisuutensa. Tällöin mallien konfiguraatio muodostaa myös mallin malleista. Nämä voivat edelleen muodostaa muita metamalleja. Ne muodostavat johdonmukaisen projektiorientoituneen kokonaisuuden jonka tuloksena on tehdasmalli.

		\subsection{Tehdasmalli}

Tehdasmalli on kuvaus joka mahdollistaa tehtaan toteutuksen, käyttöönoton ja ylläpidon.

Se on tietomalli joka kuvaa laajuudestaan riippuen tehtaan toimintaa, sen prosessia, organisaatiota, ihmisiä ja näitten aktiviteetteja. Tehdasmalli koostuu pohjimmiltaan tehdasobjekteista, niitten ominaisuuksista ja niitten välisistä relaatioista.

Tehdasmallin tulee palvella kaikkia näkökulmia joita tarvitaan.

Tietokanta on perinteisesti linkitetty tietokanta. (lähde?) Tehdasmalli voi olla dokumenttihotellin typpinen sovellus. Se voi myös olla tietokantatyyppinen tietomalli.

Tehdasmallin voidaan sanoa olevan tietomalli jonka osa automaatiojärjestelmän kuvaus on.

%%%%%%%%%%%%%%%%%%%%%%%%%%%%%%%%%%%%%
%	\clearpage
%	\section{Automaatiojärjestelmän logiikka}
	\subsection{Automaatiojärjestelmän logiikka}
%%%%%%%%%%%%%%%%%%%%%%%%%%%%%%%%%%%%%

Tehtaan automaatiojärjestelmä muodostaa oman itsenäisen osansa tehdasmallista ja asettaa omat vaatimuksensa muulle mallille.
Siinä

		\subsection{Automaatiojärjestelmän logiikan malli}

Hajautetun ohjausjärjestelmän logiikkasuunnittelulle ei ole de facto-standardia, pikemminkin joukko standardeja joita käytetään vaihtelevasti.
IEC~61131-3 ja IEC~61499 ovat yrityksia tuottaa avoimia standardeja, jotka ovat riittävän muokkautuvia mihin tahansa tarpeisiin.

PLC-ohjelmointi on vanha ja intuitiivinen ala. Vaikka olisi selviä syitä suunnittelun formalisoitumiselle

Vaikka PLC saatetaan myöhemmin ohjelmoida tikapuukaavioilla, on silti helpompaa työskennellä ja ymmärtää ohjelman rakenne logiikkadiagrammien avulla.

		\subsection{Toimintakaaviot}

Logiikkadiagrammit ovat yritys pelkistää yksinkertaisimpaan mahdolliseen esitykseen monimutkainen logiikka joka on syöttöjen ja lähtöjen välillä. Vaikka PLC saatetaan myöhemmin ohjelmoida tikapuukaavioilla, on silti helpompaa työskennellä ja ymmärtää ohjelman rakenne logiikkadiagrammien avulla. Lisäksi tehtaan käyttöönotto ja huolto helpottuvat kun
Tämän vuoksi 

Logiikkadiagrammit (mcavinew, control system documentation

Logiikkadiagrammit helpottavat tehtaan toiminnan ymmärrystä suunnittelu- ja rakennusvaiheen lisäksi tehtaan ylläpidon ja huollon yhteydessä ja ovat siksi tärkeä osa tehtaan dokumentointia.

Funktioblokit kuvaavat toisistaan erillisiä itsenäisiä laskennallisia yksiköitä ja niitten relaatioita toisiinsa. Funktioblokki voi tilastaan riippuen lähettää tapahtumia jotka vaikuttavat muitten blokkien toimintaan. Funktioblokin tekemien laskutoimitusten tulos riippuu siis sen vasemmalta puolelta tulevista syöttötiedoista. Laskennan tulokset lähetetään blokin oikealta puolelta. Toisiinsa liitetyt funktioblokit muodostavat funktioblokkiverkon. 

IEC 61499 määrittelee XML-kuvauksen funktioblokkiverkoille.

Although the IEC 61499 is not always seen as the ultimate solution, it is a useful and powerful basis for engineering tools that follow an abstract, model-driven approach and can generate the actual function blocks and their interconnections automatically from a high-level description of the required functionalities." -10.1109/MIE.2011.942175 (Sauter et. al. IEEE Industrial electronics magazine, p. 35)
PLCeitä usein ohjelmoidaan vanhasta tottumuksesta tikapuukaavioilla.
Näitten ymmärtäminen on intuitiivisesti vaikeampaa ja ne usein vaativat alan tuntemusta.


Näitten ymmärtäminen on intuitiivisesti vaikeampaa ja ne usein vaativat alan tuntemusta.

		\subsection{Muunnokset}

Hajottamalla suunnitteluprosessi kahteen voidaan prosessia kehittää ja siitä keskustella ilman syvällistä tuntemusta instrumenteista.
Kun päätökset on tehty abstraktilla tasolla, voidaan hankinnat räätälöidä sen mukaisesti ja tulos toteuttaa tehokkaasti.

Logiikkaa sisältävän järjestelmän suunnittelutyö koostuu ensin prosessin konseptualisoinnista ja toinen toteutuksesta.
Tässä työssä tarkastellaan tapaa jolla suunnittelutyön työmalli voidaan muuntaa toiseksi malliksi.

Projektin edetessä konseptista toteutukseksi huomataan, että vähintään kahdenlaisia dokumentteja tarvitaan.
Toteuttava dokumentti saattaa olla ainoa, joka näkee päivänvalon. 

Yleisin tapa tuottaa kaavioita lienee niitten käsin tekeminen.
Valmiiden grafiikkakirjastojen käyttöä on tutkinut Kääriäinen, jonka mukaan työssä .
next generation systems…

http://mv-sirius.fh-offenburg.de/ecmIC/SensorActuatorObj-Worksheet/atr_2001.pdf
%%%%%%%
%REMOVE ME LATER %
\addcontentsline{toc}{section}{Viitteet} %
\bibliography{kandi} %
\makeindex %
 \end{document} %
%%%%%%%%%%%

%%%%%%%%%%%%%%%%%%%%%%%%%%%%%%%%%%%%%%%
	\clearpage
	\section{Hierarkinen graafinpiirto}
%%%%%%%%%%%%%%%%%%%%%%%%%%%%%%%%%%%%%%%

Kerrostetut suunnatut graafit ovat luonteeltaan hierarkisia, kuten logiikkakaaviot suoritusjärjestyksensä puolesta.
Hierarkisuudella viitataan siihen, että 
Hierarkisen lähestymistavan kerrostettujen suunnattujen graafien piirtämiseen esittivät Sugiyama et. al vuonna 1981 \cite{RefWorks:9}.

Tämä lähestymistapa on tehokas ja intuitiivinen ja ollut jatkokehityksen- ja tutkimuksen kohteena \cite{RefWorks:28}.
Hierarkinen lähestymistapa koostuu kolmesta vaiheesta: kerrostamisesta, risteämisten vähentämisestä ja poikittaisasettelusta.

Lisäksi piirrettävän graafin täytyy olla etukäteen valiidisti muotoiltu.
Valiidisti muotoiltu graafi sisältää 

http://www.cs.brown.edu/~rt/gdhandbook/

Automaatiokaavioiden suoritusjärjestys on määritelty standardissa IEC 61131-3 ylhäältä alas, vasemmalta oikealle.
Suoritusjärjestysvaatimus asettaa toteutukselle rajoituksia 
Jos tätä ei oteta huomioon, saattaa graafin algoritmillinen piirto johtaa vääränlaiseen kaavioon.
Tämän vuoksi on otettu käyttöön tätä tilannetta varten tehty algoritmi.

Koska suoritusjärjestyksen eteneminen on määritelty tavalla X, voidaan käyttää algoritmia sillä rajoituksella, että Y
Automaatiokaavioiden automaattinen piirto voi tuottaa ongelmia suoritusjärjestystä mietittäessä. 

Tämä työ olettaa, että eri ohjelmien suoritusjärjestys niillä kohdin, kuin se on olennaista, on tiedossa ja merkittynä tietokantaan.

		\subsection{Syklien poisto}

Jos graafissa on syklejä, eli graafi ei ole suunnattu, käännetään syklin aiheuttavien lankojen suunta ja tehdyt muutokset merkitään muistiin.
Tätä varten on kehitetty useita erilaisia algoritmeja.



		\subsection{Kerrostus}

Jos lähdegraafi sisältää takaisinkytkentöjä, vaatii
Kerrostuksen tulee täyttää tietyt ehdot.
Sen tulee olla kompakti.


Tämä tarkoittaa että sen leveys ja pituus ovat pieniä ja kerrosten välinen etäisyys on vakio.
Graafin pituuden alaraja voidaan laskea on maksimimäärä lankoja alkunoodista loppunoodiin.
On olemassa menetelmä, jonka avulla graafin pituus voidaan minimoida leveyden kustannuksella tai siitä voidaan tehdä mahdollisimman kapea pituuden kustannuksella.

Leveyden ja pituuden minimointi taas on NP- täydellinen ongelma, seuraavasta syystä.
Jos jokainen solmi suunnatussa kerrostetussa verkossa G esittää yhden aikayksikön työtä yhdessä multiprosessorin prosessorissa, lanka (u, v) esittää rajoitetta, jonka mukaan suorituksen u täytyy edeltää suoritusta. (Battista, 273)
Tällöin voidaan löytää isomorfismi multiprosessoriajastusongelmalle, jossa tehtävänä on jakaa tehtävät prosessorien kesken siten, että tehtävä tulee suoritettua ajassa H.
Koska multiprosessoriajastusongelma on NP-täydellinen (M.R. Garey and D.S. Johnson, Computers and Intractibility: A guide to the theory of NP-completeness. 1979) , seuraa, että kerrostusongelmakin on.
Yhteys multiprosessoriongelmaan on kiinnostava käsittelemämme sovelluksen, PLC-ohjelmoinnin kannalta.
Multiprosessoriongelmaa varten tehty algoritmi, Coffman-Graham-kerrostusmenetelmä (Coffman, Optimal scheduling for two processor systems, 1972) tarjoaa ratkaisun tähän ongelmaan.
Se palauttaa graafin, jolla on jokin minimileveys, alkuperäisessä toteutuksessa prosessorien määrä, omassa tapauksessamme arkin korkeus. 
Algoritmin tavoitteena on varmistaa, että 

		\subsection{Risteyksien vähentäminen}

Risteyksien vähentäminen ei perustu solmujen tarkkoihin sijainteihin, vaan niitten keskinäiseen järjestykseen.
Ongelma on siis luonteeltaan kombinatorinen eikä geometrinen, mikä huomattavasti yksinkertaistaa ongelman ratkaisemista. 
Tästä helpotuksesta huolimatta kyseessä on NP-täydellinen ongelma (M.R. Garey ”Crossing problem is NP-complete”) siinäkin tapauksessa että kerroksia no vain kaksi.

Lähestymistapoja risteyksien vähentämiseen on vierekkäisten noodien paikkojen vaihtelu keskenään, sekä mediaanimenetelmä, jossa noodi asetetaan paikalle, joka on siihen yhtyneiden solmujen puolivälissä.

Monet tutkijat ovat Tähän se väikkäri myös! (P. Eades, ”Heuristics for Reducing crossings in 2-layered networks”, E. Mäkinen, ”experiments of drawing 2-level hierarchical graphs”, M. Jünger, ”2-layer straightline crossing minimization: Perf…” ) tutkineet eri risteysalgoritmeja. Yhtä selkeästi parasta menetelmää ei ole löytynyt. Sen sijaan paras lähestymistapa on osoittautunut algoritmeja yhdisteleväksi hybridilähestymiseksi.
(Joku et al on) saavuttanut hyviä tuloksia iteroimalla joka iteraation yhteydessä ensin uudelleenjärjestää solmut mediaanimenetelmällä, tekee paikallisia transpositioita ja lopulta pitää parhaan tuloksen seuraavaa iterointia varten.


Optimizing Automatic Layout for Data Flow Diagrams, 2011, Diploma thesis, Christian-Albrechts-Universität zu Kiel, Department of Computer Science

		\subsection{Poikittainen asettelu}

Solmujen poikittainen asettelun avulla solmujen välisten lankojen käännösten määrää pyritään vähentämään, jotta kaavion luettavuus paranee.
Risteyksien vähentämisvaiheessa saavutettu kombinatorinen ratkaisu ja sitä myötä risteysmäärä säilyy.
Poikittainen asettelu on tärkeä lopputuloksen luettavuuden kannalta.


%%%%%%%%%%%%%%%%%%%%%%%%%%%%%%%%%%%%%%%%%%%%%%%%%%%%%%
	\clearpage
	\section{Menetelmät ja esimerkki}
%%%%%%%%%%%%%%%%%%%%%%%%%%%%%%%%%%%%%%%%%%%%%%%%%%%%%5

		\subsection{Kaavioiden tuottaminen}

Käytännön toteutuksia hierarkisesta graafinpiirtomenetelmästä on useita (viittaa: graphviz, tulip, kieler)’
An open graph visualization system and its applications to software engineering (2000) 

Automatic generation of PLC automation projects

from component-based models
Elisabet Estévez
Käytännönläheinen selostus siitä, kuinka aiempaa teoriaa sovelletaan.

		\subsection{Kaavioiden tuottaminen}
			\subsubsection{Vertaillut vaihtoehdot}
	
Funktioblokkien editointiin on olemassa useita työkaluja.
Rockwell Automationin ilmainen funktioblokkieditori, FBDK on tutkimuksessa laajassa käytössä.
FBDK ei kuitenkaan tarjoa tapoja muuttaa kaavion asettelua algoritmillisesti.
FBench on avoimen lähdekoodin standardimukainen ohjelmisto, joka kuitenkin kärsii samoista puutteista kuin FBDK.
PROFACTORin kehittämä 4DIAC tarjoaa standardeja noudattavan funktioblokkieditorin.
Se ei kuitenkaan mahdollista blokkien välisten lankojen reitittämistä.
ISaGRAF ja NxtControl ovat kaupallisesti tuotettuja ohjelmistoja joita ei valitettavasti voitu työtä varten arvioida.

			\subsubsection{KIELER}

Koeohjelmistoksi valittiin KIELER, Kielin yliopistossa kehitetty yleiseen graafinpiirtoon tuotettu ohjelmisto, jolle on kehitetty IEC 61499-standardin mukaisten funktioblokkidiagrammien piirto-ohjelmisto CAKeFEED. 
Kerrotaan tarkemmin ohjelmistosta jolla koe on suoritettu

		\subsection{Tulosten arviointiperusteet}

Lankojen risteämismäärien minimointi !
Ortogonaalisuus!
Graph Layout Aesthetics in UML Diagrams: User Preferences

The Aesthetics of Graph Visualization

	\clearpage
	\section{Tulokset}
	
Olemme osoittaneet yhteyden graafiteorian suunnatuilla puilla ja logiikkadiagrammeilla ja näyttäneet miten graafipiirtoalgoritmeja voidaan käyttää automaatiokaavioiden generointiin.
Tapa X on kivempi kuin tapa Y.

	\section{Yhteenveto ja tulevaisuudennäkymät}
	
Siistejä juttuja, joo. Ollaan saavutettu tavoite A. Tässä hommassa kokonaisuutena on vielä haasteita. Olisipa hienoa saada selville asia Z

Vaikka automaatiokaavioiden tuottamisen automatisointi onnistuttaisiin tekemään, on se vain yksi viidestä erilaisesta suunnitelunäkymästä.

Automaatiosuunnittelu on siirtynyt yhä enemmän tietopohjaisesta tutkimuksesta

RDF:ää voidaan käyttää tehtaan PI-kaavion miettimiseen.
Graph theory - An RDF model is a labeled, directed multi-graph


\clearpage
\addcontentsline{toc}{section}{Viitteet}

\bibliography{kandi}
\makeindex
\end{document}
