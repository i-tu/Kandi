	\documentclass[finnish,12pt]{article}
	\usepackage{aaltothesis}
	
      % \usepackage[T1]{inputenc}


	%% Use this if you run latex and use eps-format pictures
	%\usepackage[dvips]{graphicx}
	\usepackage[pdftex]{graphicx} 

	\pdfoptionalwaysusepdfpagebox=5
	
	\makeindex
	\usepackage[pdfpagemode=None,colorlinks=false,urlcolor=red,linkcolor=blue,citecolor=black,pdfstartview=FitH]{hyperref}

	%% Use this if you do not like hyperref package - this defines url environment and formats it correctly
	%\usepackage{url}

	% Math stuff
	\usepackage{amsfonts,amssymb,amsbsy}

	\usepackage{setspace}
	\onehalfspacing
	\setlength\parindent{0pt}
	\usepackage[parfill]{parskip}
	
	\usepackage{lmodern}
	
	% Layout
	\setlength{\hoffset}{-1in}
	\setlength{\oddsidemargin}{35mm}
	\setlength{\evensidemargin}{25mm}
	\setlength{\textwidth}{15cm}
	\setlength{\voffset}{-1in}
	\setlength{\headsep}{7mm}
	\setlength{\headheight}{1em}
	\setlength{\topmargin}{25mm-\headheight-\headsep}
	\setlength{\textheight}{23cm}

	\bibliographystyle{unsrt}

	\uselogo{blue}{?}{tkk}
	\university{}{Aalto-yliopisto}
	\school{}{Sähkötekniikan korkeakoulu}
	\faculty{}{Automaatio- ja systeemitekniikan laitos}
	\degreeprogram{}{Automaatio- ja systeemitekniikka}
	\univdegree{BSc}
	\author{Ian Tuomi}{}
	\thesistitle{}{Teollisen ohjausjärjestelmän logiikkakuvauksen muuntaminen esteettiseksi kaavioksi}

	\place{Espoo}
	\date{6.12.2012}
	\supervisor{DI Mika Str\"{o}mman}{DI Marek Matusiak}
	\instructor{}{DI Mika Str\"{o}mman}

	\keywords{Automaatiojärjestelmä, Automatisoitu suunnittelu, ohjelmoitava logiikka, PLC, IEC 61131, FBD, graafinpiirto, suunnattu graafi}

	\begin{document}

	 \makecoverpage

	\begin{abstractpage}[finnish]

Työssä esitellään automaatiosuunnitteluun liittyviä malleja, kuvauksia ja metamalleja.

Esteettisen kaavion tunnusmerkit määritellään, ja esitellään menetelmä, jonka tuloksena sellainen voidaan laatia.
Menetelmän vaiheet ja niihin liittyviä algoritmeja esitellään.

	\end{abstractpage}

	\tableofcontents

\addcontentsline{toc}{section}{Sisällysluettelo}


%% Käsitteet
	\mysection{Lyhenteet}

	\begin{tabular}{ll}	     	    
DCS	& Distributed Control System \\
	& Hajautettu ohjausjärjestelmä \\ \\
DFS	& Depth-first search\\
	& tapa kulkea graafin läpi \\\\

FBD	& Function Block Diagram\\
	& IEC 61131-3 standardin määrittelemä ohjelmointikieli\\\\
IEC	& International Electrotechnical Commission,\\
	& Kansainvälinen sähköalan standarointiorganisaatio\\\\
NP & Nondeterministic Polynomial time\\
	& Epädeterministisellä Turingin koneella polynomiaalisessa\\&ajassa ratkeavien ongelmien joukko \\\\ 
PLC	& Programmable Logic Circuit \\
	& Ohjelmoitava logiikkapiiri\\\\
XML	& Extensible Markup Language\\
	& Merkintäkieli johon voidaan sisällyttää ja jolla voidaan jäsentää tietoa  \\\\
XSL	& Extensible Stylesheet Language\\
	& Kieliperhe, jolla voidaan määritellä XML-formaatteja tai tehdä XML-muunnoksia \\\\

\end{tabular}

	\cleardoublepage
	\storeinipagenumber
	\pagenumbering{arabic}
	\setcounter{page}{1}

% Kandi %

	\section{Johdanto}
	\thispagestyle{empty}

Teollista automaatiojärjestelmää suunnitteltaessa tuotetaan suuri määrä erilaisia dokumentteja.
Ne yhdessä muodostavat kuvauksen, joka mahdollistaa järjestelmän toteutuksen, käyttöönoton ja ylläpidon.
Dokumentaation laatiminen käsin on suurissa projekteissa työlästä ja tehotonta.
Nykyaikaisessa automaatio- ja instrumentointisuunnittelussa ongelma korostuu \cite{RefWorks:41}.
Ratkaisuista on tullut monimutkaisempia, ja laatu- ja joustavuusvaatimukset ovat lisääntyneet.
Lisäksi vaaditut toteutusajat ovat jatkuvasti lyhentyneet.

Johtuen suunnitteluprosessiin kohdistuneista vaatimuksista, työtä on pyritty tehostamaan automatisoiduilla apuvälineillä.
Metodiikka ei kuitenkaan ole ilmeisistä standardoimiseduista huolimatta yhtenäistynyt.
Automaatiosuunnittelun tutkimuskenttä on lisäksi hajanainen, ja yhteistyö on vähäistä.
Vaikka lähestymistavat ovat vaihtelevia, suunnittelussa on nähtävissä yhtenäinen siirtymä kohti jaettuja tietomalleja.
Automaatiojärjestelmien näkökulmasta jaettujen tietomallien käyttö tarkoittaa,
että suunnittelujärjestelmään tallentuu projektin edetessä kaikki sähköistys ja instrumentointi automaatiojärjestelmineen ja kenttälaitteineen.
Erilaiset kaaviot voidaan usein tuottaa suoraan tietomallin sisältävästä tietokannasta.

Tietomalliin perustuva lähestymistapa vaatii sen laajuuden ja rajoitukset määrittelevän metamallin.
Se ohjaa ja määrittää suunnittelun kohteen lisäksi myös yleisesti suunnittelutöiden kulkua.

Tämän kandidaatintyön kohteena on ohjausjärjestelmän logiikan muuntaminen standardeja noudattavaksi kaavioksi.
Logiikkakaavio on tapa pelkistää järjestelmän eri osien syöttöjen ja lähtöjen välillä vallitseva logiikka visuaaliseen ja helposti ymmärrettävään muotoon.
Se on määritelty yksiselitteisesti ja on tulkittavissa suunnatuksi graafiksi, jolloin algoritmillisen graafinpiirron tuloksia voidaan hyödyntää.

Algoritmillinen graafinasettelu on perusteellisesti tutkittu ala, jonka tuloksia voidaan käyttää kaikenlaisiin suunnattuihin graafeihin.
Tässä työssä eritellään Sugiyaman ym. \cite{RefWorks:9} esittämään hierarkiseen lähestymistapaan perustuva, logiikkakaavioiden kaltaisten tietovirtakaavioiden asetteluun soveltuva algoritmi.

Työssä on myös kokeellinen osio, jossa laadittua algoritmia käytetään konkreettisen kaavion asetteluun.
Lopuksi pohditaan tapoja sisällyttää menetelmä käytännön suunnittelutyöhön.

	\clearpage
	\section{Tehtaan mallit}

Tehdasmalli on kuvaus, joka mahdollistaa tehtaan toteutuksen, käyttöönoton ja ylläpidon.
Se on tietomalli, joka kuvaa tehtaan toimintaa, sen prosessia, organisaatiota, ihmisiä ja näitten aktiviteetteja.
Pohjimmiltaan se koostuu tehdasobjekteista ominaisuuksineen ja niitten välisistä relaatioista. \cite{RefWorks:41}

Suunnittelijoilla on työtehtävistään erilaisia tarpeita tehdasmallin suhteen.
Työn kohde on monitahoinen, eikä mikään yksi esitystapa pysty kattamaan kaikkia suunniteltavan järjestelmän näkymiä.
Tehdasta tulee kuvata useilta eri näkökannoilta erilaisilla tarkkuuksilla kunkin näkymän vaatimuksien mukaisesti.
Tehdasmalliin kuuluu siis useita malleja, jotka yhdessä muodostavat kokonaismallin.

Jokaisella mallilla on myös metamalli. Se määrittelee millaisia objekteja se voi sisältää, mitä ominaisuuksia sillä voi tai täytyy olla ja millaisia yhteyksiä objektien välillä on.
Metamallin loogiset riippuvuudet määrittelevät suunnittelun työtapoja ja järjestystä.
Ne kuvaavat millaiset tietomallit ovat suunnittelujärjestelmän kannalta hyväksyttäviä tehdasmalleja ja
määräävät sen laajuuden. Metamallin määritteleminen tekee työn tuloksesta ennakoitavan ja mahdollistaa tehokkaiden apuvälineiden kehittämisen.
Määrittelyllä myös suljetaan pois eriäviä tapoja kuvata tehdasta ja mahdollistetaan määrittelyyn perustuvat työkalut ja suunnittelukäytännöt.
Tällaisia ovat esimerkiksi uudelleenkäytettävää koodia sisältävät kirjastot ja erilaiset mallimuunnokset.

Suunnittelussa käytettävät metamallit valitaan siten, etteivät ne ole ristiriidassa ja muodostavat yhdessä toisiaan täydentävän kokonaisuuden.
Tällöin metamallien konfiguraatio muodostaa itsessään mallin.


	\subsection{Automaatiojärjestelmän kuvaukset}

Teollisuuden hajautettujen ohjausjärjestelmien suunnittelussa on havaittu
tarpeellisiksi ainakin kolme kuvausta: toiminnallinen, fyysinen sekä ohjelmistollinen \cite{RefWorks:38}.

		\subsubsection{Toiminnallinen kuvaus}

Toiminnallinen kuvaus määrittelee säätöjärjestelmän toiminnan tavalla, joka toteutettuna täyttää sille asetetut käyttäjävaatimukset.
Se kuvaa järjestelmän toimintaa, muttei määrää sen tarkkaa teknistä toteutusta. \cite{RefWorks:60}

Tätä näkymää määrittäessä voivat eri alojen suunnittelijat osallistua työhön ilman
tuntemusta instrumentoinnista tai ohjelmistokehityksestä. Kun päätökset on tehty
abstraktilla tasolla, voidaan hankinnat räätälöidä sen mukaisesti ja tulos
toteuttaa tehokkaasti.
Tällaisen määrittelyn avulla on mahdollista käyttää formaaleja verifiointimenetelmiä joitten avulla
voidaan osoittaa, että järjestelmän suunnitelma toteuttaa sille asetetut vaatimukset.
Näin suunnitelman oikeellisuus voidaan osoittaa ennen investointeja toteutukseen. \cite{RefWorks:41}

		\subsubsection{Fyysinen näkymä}

Fyysiseen näkymään suunnitellaan järjestelmän johdotukset, kaapelit, kenttäväylät,
ohjauskeskukset, toimilaitteiden sijainnit sekä ylipäänsä kaikki mitä vaaditaan 
järjestelmän fyysiseen käyttöönottoon. Sen käsittely ei kuulu tämän työn laajuteen.

		\subsubsection{Ohjelmistonäkymä}

Ohjelmistonäkymä kattaa funktionaalisen suunnittelun toteutuksen ohjelmamuodossa.
Asiakkaalla on yleensä omat toiveensa sen suhteen, millaiseen muotoon ohjelma tehdään.
Ohjelmiston toteuttamiseen ei pitäisi liittyä enää suunnittelutyötä - toisin sanoen toiminnallisen kuvauksen määrittelyjen täytyy olla riittävän tarkkoja jotta niitten mukaan tehdyssä ohjelmassa ei ole sen toimintaan vaikuttavia tulkinnanvaraisuuksia.

Ohjelmistototeutuksen työtavat jäävät usein määrittelemättömiksi ja muusta suunnittelutyöstä irrallisiksi,
vaikka pyrkimyksiä toteutustapojen nitten integroimiseksi suunnitteluprosessiin entistä vahvemmin on ollut.
Toteutustapa jää yksittäisten suunnittelijoiden määriteltäväksi, mistä seuraa että yhtenäistä työtapaa harvoin on.
Toteutetun ohjelmakoodin uudelleenkäyttö jää myös tyypillisesti vähäiselle asteelle. \cite{RefWorks:42}


	\subsection{Logiikan kuvaukset}

%McAvinew ym. \cite{RefWorks:53}
%Meier, automation body of knowledge: \cite{RefWorks:54}
%http://www.geocities.ws/andrikuncoro/Instrumentation/1811_06.pdf

Logiikan kuvaukset helpottavat tehtaan toiminnan ymmärrystä suunnittelu- ja rakennusvaiheen lisäksi tehtaan n ja huollon yhteydessä ja ovat siksi tärkeä osa tehtaan lopullista dokumentaatiota.
Teollisessa automaatiosuunnittelussa sallitut logiikan esitystavat on määritelty erityisesti selkeyttä, turvallisuutta ja ennustettavaa toimintaa silmälläpitäen.
Joskus loogista järjestelmää voidaan kuvailla kertomalla esimerkinomaisesti sen toiminnasta eri olosuhteissa.
Tällainen lähestymistapa ei kuitenkaan yleensä ole riittävä.
Lisäksi määritellään yleensä ainakin muuttujia, säätöpiirejä, sekvenssejä sekä niitten välisiä kytkentöjä.\cite{RefWorks:41}

Jatkuva prosessiohjaus voidaan esittää selkeästi järjestelmän PI-kaavioissa, mutta diskreetin ohjauksen esittämiseen tarvitaan erilaisia esityksiä, kuten logiikkadiagrammeja. \cite{RefWorks:54}

Logiikkadiagrammit kuvaavat syöttöjen ja lähtöjen välistä logiikkaa.
Määrittelyn mukaisia kaavioita ei kuitenkaan ole välttämättä helppoa laatia.

		\subsubsection{Toimintakaaviot}

Toimintakaavio on funktionaalisen näkymän toteutus, joka tavallisesti valmistellaan järjestelmän suunnittelun varhaisessa vaiheessa prosessikaaviosta.
Se määrittelee ohjausjärjestelmän toiminnot ja sitä pidetään ajantasaisena järjestelmän suunnittelun edetessä.
Se toimii suunnittelun apuna ja on lopulta osa lopullisen järjestelmän ohjeistusta.

Nykyaikaisessa automaation ohjelmistosuunnittelussa käytetään usein ohjelmointikieliä jotka ovat muodoltaan lähellä toiminnallisia logiikan kuvauksia ja dokumentoivat itsensä hyvin.
T'ämän vuoksi toiminnallinen kuvaus usein jätetään tekemättä, jolloin syntyy säästöjä. \cite{RefWorks:41}

		\subsubsection{Ohjelmakoodi}

Ohjelmakoodi on ohjelmanäkymän toteutus.
IEC~61131-3 on laajassa käytössä oleva avoin ohjelmakoodin standardi joka määrittelee automaatiologiikan ohjelmointimenetelmät ja pyrkii olemaan riittävän joustava mihin tahansa tarkoitukseen. \cite{RefWorks:62}
Se sisältää viisi erilaista ohjelmointikieltä, joista tässä työssä käsitellään erityisesti FBD:tä.

FBD-kieli koostuu toimilohkoista ja niiden välisistä langoista.
Toimilohkot kuvaavat toisistaan erillisiä itsenäisiä laskennallisia yksiköitä.
Langat kuvaavat niitten relaatioita toisiinsa.
Toimilohko voi tilastaan riippuen lähettää tapahtumia jotka vaikuttavat muitten toimilohkojen toimintaan.
Sen tekemien laskutoimitusten tulos riippuu sen vasemmalta puolelta tulevista syöttötiedoista ja sen suorittaman laskennan tulokset lähetetään blokin oikealta puolelta.
Toisiinsa liitetyt toimilohkot muodostavat verkon, joka määrittelee laajemman toiminnallisuuden. \cite{RefWorks:55}

Toimilohkojen muokkaamiseen on olemassa useita työkaluja.
Näistä mainitsemisen arvoisia ovat ainakin FBDK, 4DIAC ISaGRAF ja NxtControl.
Ne eivät kuitenkaan tarjoa kaavioiden asetteluun kuin yksinkertaisia työkaluja, jos ollenkaan.

Automaatiokaavioiden suoritusjärjestys etenee standardin IEC 61131-3 mukaan vasemmalta oikealle.
Lisäksi toimilohkoa ei ajeta ennen kuin kaikki sitä edeltävät toimilohkot on suoritettu.
Keskinäisen suoritusjärjestyksen täytyy myös tulla esille toteuttavassa ohjelmistossa.
Suoritusjärjestys kulkee usein ylhäältä alas, mutta tätä ei voi olettaa aina todeksi. \cite{RefWorks:62}

Suoritusjärjestysvaatimus asettaa seuraavassa kappaleessa tarkasteltavalle graafinpiirrolle rajoituksia.
Jos niitä ei oteta huomioon, saattaa graafin algoritmillinen piirto johtaa vääränlaiseen kaavioon.
Mahdollisten takaisinkytkentöjen tapauksessa vaaditaan huolellisuutta, jottei syklien poistovaiheessa kaavion järjestys muutu.


		\subsubsection{Logiikan tietokantakuvaukset}

Suunniteltu logiikka tallennetaan jaettuun suunnittelumalliin, johon kaikilla järjestelmän suunnittelijoilla on yhteys.
Tapa jolla logiikka tallennetaan täytyy olla yhdenmukainen, jotta suunnittelutyössä voidaan tehdä tehokasta yhteistyötä työpaikalla jaettujen yhtenäisten suunnittelun apuvälineiden avulla.
Lisäksi kuvauksen tulee olla formaali jotta logiikkaa voitaisiin käsitellä algoritmillisesti.
Formaalilla tarkoitetaan tietynlaista, tarkkaan määritetyn syntaksin mukaista kuvausta.
Tällöin tulee myös mahdolliseksi toteuttaa helposti tietoa käsitteleviä muunnosalgoritmeja.
Muunnoksella tarkoitetaan jonkin tiedon kuvauksen muuttamista toiseksi kuvaukseksi.

Standardin IEC 61131-3 mukaisten ohjelmointikielten kuvaamiseksi on ehdotettu PLCopen-standardia.
PLCopen -standardin mukainen logiikan XML-esitys voi sisältää toimilohkon toiminnallisuuden kannalta välttämättömien asioiden lisäksi toimilohkon sijainnin, koon ja jopa lankojen reititykset.
PLCopen -standardin mukainen esitys voidaan tällöin laatia siten, että se on tuotettavan kaavion asettelun kannalta täysin yksiselitteinen.
Tällaista esitystä voidaan silloin pitää dokumentaation lopullisena muotona. \cite{RefWorks:64}

Kun logiikkakuvauksen muoto on määritelty tarkasti, voidaan määritelmän mukaisesti laadittu logiikkakuvaus muuntaa määritelmään nojaavien sääntöjen perusteella.
Monet automaatiologiikan suunnitteluohjelmistot pystyvät muuntamaan käyttämänsä kuvauksen PLCopen-muotoiseksi tiedoksi, jolloin suunnittelutietoa voidaan siirtää eri järjestelmien kesken.
Erilaisia XML-kuvauksia on helppoa laatia ja niitten sisältämien tietojen pohjalta voidaan tuottaa minkä tahansa muotoisia dokumentteja käyttäen hyväksi XSL-muunnoksia \cite{RefWorks:61}. Eri XML-pohjaisia tiedostoformaatteja käyttävien suunnitteluvälineiden välinen integraatio on tällöin helposti toteutettavissa.

Olemassaolevien suunnitteluohjelmistojen varaan rakennettu suunnittelujärjestelmä voi kuitenkin kohdata haasteita, jos luotetaan liikaa ohjelmien standardinmukaisuuteen.
Valmistajat jättävät usein standardinmukaisiksi kutsutuista ohjelmistoista määrittelyn mukaisia ominaisuuksia toteuttamatta.
Tämä siitä, että ohjelma on määritelty standardinmukaiseksi, jos se toteuttaa siitä jonkin osan ja mainitsee mitä se jättää toteuttamatta. \cite{RefWorks:42}

	\clearpage

	\section{Yhteenveto}

Teollisen automaatiojärjestelmän suunnittelu- ja toteutusprosessi on monimutkainen, kallis ja pitkä.
Siinä käytettäviä työvälineitä tehostamalla työ nopeutuu, ja voidaan saavuttaa merkittäviä säästöjä.

Työvälineiden tehostamista on akateemisissa julkaisuissa kartoitettu runsaasti viime vuosina.
Tutkimukset ovat kuitenkin olleet usein turhan abstrakteja ja vieraannuttavia teollisille toimijoille.
Tämä työ on pyrkinyt tarjoamaan ratkaisun, jolla suunnittelutyötä voidaan konkreettisesti tehostaa, ja joka voidaan helposti integroida olemassaolevaan järjestelmään.

Työssä on osoitettu yhteys suunnattujen graafien ja logiikkadiagrammien välillä, sekä todettu niitten asetteluun kehitettyjen algoritmien käyttökelpoisuus teollisuuden automaatiosuunnittelussa.
Esitelty algoritmi tuottaa hyviä tuloksia, ja on sovellettavissa edelleen erilaisiin tarpeisiin.

On huomattava määrä hyviäkin ideoita, joita ei toteuteta.
Näin tulisikin asian olla, sillä vaikka idea olisi hyvä ja käytännöllinen, se ei välttämättä ole kannattava.
Tätä 
Lisäksi tarvitaan myös toimijoiden tahtotila.

Yritysten tulee pyrkiä kohti entistä integroidumpaa ja ketterää suunnitteluprosessia, jotta on mahdollista kilpailla palvelun nopeudella ja hinnalla.

Kaavioasettelun automatisoinnin myötä nousee kysymys, voidaanko automatisoidun koodigeneroinnin menetelmillä luoda automaatiojärjestelmien suunnitelmia korkeamman tason kuvauksilla, joista matalemman tason kuvaukset tuotetaan.

\clearpage
\addcontentsline{toc}{section}{Viitteet}

\bibliography{kandi}

\end{document}